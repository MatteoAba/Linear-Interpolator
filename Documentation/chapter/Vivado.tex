\section{Vivado Report}

In this last section an \textbf{automatic synthesis} will be done, through the tool \textbf{Xilinx Vivado}. The target working device is the FPGA \textit{xc7z010clg400-1} and the three steps that will be analyzed are the following:

\begin{enumerate}
    \item \textbf{Elaborated Design} Analysis.
    \item \textbf{Synthesis} and Report analysis.
    \item \textbf{Implementation} and Report analysis.
\end{enumerate}

\subsection{Elaborated Design Analysis}

In this first part will be analyzed the \textbf{Elaborated Design}, that is the Vivado representation of the Linear Interpolator at the \textbf{Register Transfer Level}. It should be equal to the structure defined in the  Architecture Description Section.

\begin{figure}[H]
    \centering
    \includegraphics[width=1\textwidth]{img/Chapter5/Elaborated.png}
    \caption{Elaborated Design}
    \label{fig:ED}
\end{figure}

From the figure above is clear that the circuit respects the structure chosen in the planning stage, so it's possible to start the synthesis.

\subsection{Synthesis Analysis}

In this phase the Elaborated Design previously generated will be \textbf{translated} in circuits that the FPGA \textbf{can implement}. This will be done respecting all given \textbf{constraints}. In this case the only constraint is the clock, that must have a period of $8ns$.

The synthesis result is the following:

\begin{figure}[H]
    \centering
    \includegraphics[width=0.5\textwidth]{img/Chapter5/SyntesisResult.png}
    \caption{Synthesis Result}
    \label{fig:SR}
\end{figure}

There are no errors or warnings, so it's possible to study the reports.

\subsubsection{Timing Report and Critical Path}

The \textbf{timings} obtained are as follows:

\begin{figure}[H]
    \centering
    \includegraphics[width=1\textwidth]{img/Chapter5/SyntesisTiming.png}
    \caption{Timing Report}
    \label{fig:STR}
\end{figure}

First of all, there are \textbf{no negative slacks}, so the clock with a period of $8ns$ is clearly enough for the produced design. Moreover, the \textbf{worst negative slack} is $0.986ns$, so the clock can be at least faster of this value.

Analyzing the \textbf{synthesized design}, the \textbf{critical path} is the following:

\begin{figure}[H]
    \centering
    \includegraphics[width=1\textwidth]{img/Chapter5/SyntesisSetupCrit.png}
    \caption{Critical Path for Setup time}
    \label{fig:SCPS}
\end{figure}

\subsubsection{Utilization Analysis}

The \textbf{utilization report} obtained are as follows:

\begin{figure}[H]
    \centering
    \includegraphics[width=0.6\textwidth]{img/Chapter5/SyntesisUtilization.png}
    \caption{Utilization Report}
    \label{fig:SU}
\end{figure}

The utilization for \textbf{Look Up Tables} and \textbf{Flip Flops} are lower then the $1\%$, so the FPGA has all resources needed to implement this circuit.

\subsubsection{Power Consumption Analysis}

The last report is about the \textbf{power consumption}. This is a very rough estimation, but allows to have a general idea of the circuit's power requirements.

\begin{figure}[H]
    \centering
    \includegraphics[width=1\textwidth]{img/Chapter5/SyntesisPower.png}
    \caption{Power Report}
    \label{fig:SPR}
\end{figure}

The \textbf{general consumption} is lower than  $100mW$. Moreover, it's clear that the percentage of static power is much greater than the dynamic power. So it's possible to say that the \textbf{switching activity} in this circuit is relatively low.

\subsection{Implementation analysis}

In this phase the \textbf{place and route} will be performed by Vivado, with some useful \textbf{optimizations}. In general this phase is preceded by the \textbf{I/O Planning}, in which the I/O Physical Ports of the FPGA are associated with the I/O Ports of the Elaborated Design. But the FPGA has not enough phisical ports to implement an input and an output of 16 bit. So the implementation will be performed in \textit{Out of Context Mode}, that allows to do implementation without I/O Planning. 

The implementation result is the following:

\begin{figure}[H]
    \centering
    \includegraphics[width=0.5\textwidth]{img/Chapter5/ImplementationResult.png}
    \caption{Implementation Result}
    \label{fig:IR}
\end{figure}

There are no errors, but there are warnings. They will be analyzed later.

\subsubsection{Timing Report and Critical Path}

The \textbf{timings} obtained are as follows:

\begin{figure}[H]
    \centering
    \includegraphics[width=1\textwidth]{img/Chapter5/ImplementationTiming.png}
    \caption{Timing Report}
    \label{fig:ITR}
\end{figure}

Even in this case there are \textbf{no negative slacks}, so a clock period of $8ns$ is fast enough. But in this case the \textbf{worst negative slack is worse} than before. This is due to the fact that the implementation allows to to obtain a more accurate estimate.

The \textbf{critical path} that cause this slack is the following:

\begin{figure}[H]
    \centering
    \includegraphics[width=1\textwidth]{img/Chapter5/ImplementationSetupCrit.png}
    \caption{Critical Path for Setup time}
    \label{fig:ICPS}
\end{figure}

\subsubsection{Utilization Analysis}

The \textbf{utilization report} obtained are as follows:

\begin{figure}[H]
    \centering
    \includegraphics[width=0.6\textwidth]{img/Chapter5/ImplementationUtilization.png}
    \caption{Utilization Report}
    \label{fig:IU}
\end{figure}

Compared to the synthesis, the number of \textbf{Look Up Table} has increased to $110$, while the \textbf{Flip Flops} are the same.

\subsubsection{Power Consumption Analysis}

As regards power consumption, the results are the following:

\begin{figure}[H]
    \centering
    \includegraphics[width=1\textwidth]{img/Chapter5/ImplementationPower.png}
    \caption{Power Report}
    \label{fig:IPR}
\end{figure}

Compared to the synthesis, there are no changes.

\subsubsection{Warnings Analysis}

The Implementation's Warnings are the following:

\begin{figure}[H]
    \centering
    \includegraphics[width=1\textwidth]{img/Chapter5/ImplementationWarning.png}
    \caption{Implementation Warnings}
    \label{fig:IW}
\end{figure}

All of the errors refer to Vivado's impossibility to \textbf{place} the \textbf{clock source}. 


With engineer Nannipieri we verified that at the end of the implementation in check timing the \textbf{no\_clock count} was \textbf{zero}:

\begin{figure}[H]
    \centering
    \includegraphics[width=0.6\textwidth]{img/Chapter5/NoClockCheck.png}
    \caption{Check Timing Report}
    \label{fig:NCC}
\end{figure}

So we concluded that these are \textbf{internal vivado warnings}, which in the case of using the interpolator within a more complex design should disappear.

Having input and output of the circuit at only 16 bits, is possible to do a test with an implementation \textbf{not} in \textbf{out of context mode} (thus letting vivado do \textbf{automatic routing}).

As can be seen in the figures \ref{fig:WR3} and \ref{fig:TR3}, the warnings are gone and the timings are similar to the previous implementation:

\begin{figure}[H]
    \centering
    \includegraphics[width=0.6\textwidth]{img/Chapter5/ImplementationAutomatic.png}
    \caption{Warnings Report}
    \label{fig:WR3}
\end{figure}

\begin{figure}[H]
    \centering
    \includegraphics[width=1\textwidth]{img/Chapter5/ImplementationAutomaticResults.png}
    \caption{Timing Report}
    \label{fig:TR3}
\end{figure}